\sectioncentered*{Заключение}
\addcontentsline{toc}{section}{Заключение}

Предметной областью данного дипломного проекта является процесс проведения многопользовательских викторин для участников: игроков и организатора. 
Был проведен поиск существующих программные средств этого рода, по его результатам был сделан вывод о существовании частичных аналогов. 
Были выявлены недостатки существующих решений. Было предложено программное средство, которое должно усовершенствовать процесс проведения многопользовательских викторин.

На основании проведенного анализа предметной области были выдвинуты требования к программному средству. 
В качестве технологий разработки были выбраны наиболее современные существующие на данный момент средства, широко применяемые в индустрии. 
Спроектированное программное средство было успешно протестировано на соответствие спецификации функциональных требований. 
Исходя из анализа предметной области, а также фактов наличия множества недостатков в аналогах позволили сделать вывод о 
целесообразности проектирования и разработки программной системы. Результаты, полученные в ходе выполнения технико-экономического обоснования только подтвердили данный вывод.

Разработано программное средство проведения многопользовательских викторин, целевой платформой которого является настольные компьютеры и которое поддерживает следующие функции:
\begin{itemize}
	\item создание и управление пакетами с вопросами;
	\item разделение пользователей на игроков и организаторов с поддержкой анонимной авторизации.
	\item создание, изменение, удаление игровых лобби;
	\item использование сервиса-координатора для установки соединения между клиентами;
	\item несколько видов вопросов повышающих уровень удовлетворения игровым процессом;
	\item система контроля ведущего за игровым процессом;
	\item автоматическое определение победителя в конце игры.
\end{itemize}

Для разработки программы по созданию пакетов с вопросами был использован WPF, а для сохранения результатов на диск использовалась бинарная сериализация. В качестве технологии
для разработки сервиса-координатора использовался .NET Core, в свою очередь координатор работает через Docker. Для разработки клиента был использован
WPF, для общения по сети между клиентами использовался протокол TCP/IP.
