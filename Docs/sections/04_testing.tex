\section{Тестирование и проверка работоспособности программного средства}
\label{sec:testing}

Тестирование программного обеспечения – процесс анализа программного средства и сопутствующей документации с целью выявления дефектов и
повышения качества продукта \cite{testing_ref}. Вот уже несколько десятков лет его стабильно включают в планы разработки как одна из основных работ, причем
выполняемая практически на всех этапах проектов. Важность своевременного выявления дефектов подчеркивается выявленной эмпирически зависимостью
между временем допущения ошибки и стоимостью ее исправления: график данной функции круто возрастает.

Тестирование можно классифицировать по очень большому количеству
признаков. Основные виды классификации включают следующие \cite{testing_ref}:

\begin{enumerate}
	\item по запуску кода на исполнение:
	\begin{enumerate}
		\item статическое тестирование – без запуска программного средства;
		\item динамическое тестирование – с запуском;
	\end{enumerate}
	\item по степени автоматизации:
	\begin{enumerate}
		\item ручное тестирование – тестовые случаи выполняет человек;
		\item автоматизированное тестирование – тестовые случаи частично или полностью выполняет специальное инструментальное средство;
	\end{enumerate}
    \item по принципам работы с приложением:
	\begin{enumerate}
		\item позитивное тестирование – все действия с приложением выполняются строго в соответствии с требованиями без недопустимых действий или некорректных данных;
		\item негативное тестирование – проверяется способность приложения продолжать работу в критических ситуациях недопустимых действий или данных;
	\end{enumerate}
\end{enumerate}

В данном разделе проведем динамическое ручное тестирование. В таблице~\ref{table:testing:positive} приведен список тестовых случаев, относящихся к позитивному тестированию, в таблице~\ref{table:testing:negative} -- к негативному.

% Зачем: свой счетчик для нумерации тестов.
\newcounter{testnumber}
\newcommand\testnumber{\stepcounter{testnumber}\arabic{testnumber}}

% Переключаем команды нумерации для шагов тестов. В конце файла вернем всё как было.
\renewcommand{\labelenumi}{\arabic{enumi})}
\renewcommand{\labelenumii}{\asbuk{enumii})}

\begin{landscape}
	\begin{longtable}[l]{|>{\centering}m{0.15\textwidth}
					  |p{0.8\textwidth}
					  |p{0.34\textwidth}
					  |>{\centering\arraybackslash}m{0.16\textwidth}|} 
	\caption{Тестовые случаи позитивного тестирования}
	\label{table:testing:positive}\\

	\hline
	\begin{minipage}{1\linewidth}
		\centering Модуль (экран)
	\end{minipage} & 
	\begin{minipage}{1\linewidth}
		\centering Описание тестового случая
	\end{minipage} & 
	\begin{minipage}{1\linewidth}
		\centering Ожидаемые результаты
	\end{minipage} & 
	\centering\arraybackslash Тестовый случай пройден? \endfirsthead

	\caption*{Продолжение таблицы \ref{table:testing:positive}}\\\hline
	\centering 1 & \centering 2 & \centering 3 & \centering\arraybackslash 4 \\\hline \endhead

	\hline
	\centering 1 & \centering 2 & \centering 3 & \centering\arraybackslash 4 \\

    \hline
	Авториза-ция &
	\begin{minipage}[t]{1\linewidth}
		\vspace{-1\baselineskip}
		\testnumber. \textbf{Авторизация}.\newline
 		Предусловие: не иметь авторизации.
 		\begin{enumerate}
 			\item Ввести имя пользователя.
 			\item Нажать кнопку <<Use credentials>>.
 		\end{enumerate}
 	\end{minipage} &
	\begin{minipage}[t]{1\linewidth}
		\vspace{-1\baselineskip}
		\begin{enumerate}
			\item Кнопка использования данных становится активной.
			\item Отображается главное меню. Над главным меню отображается выбранный аватар и имя пользователя.
		\end{enumerate}
	\end{minipage} \newline &
	\begin{minipage}[t]{1\linewidth}
		\vspace{-1\baselineskip}
		\centering Да
	\end{minipage} \\
	\hline

	Главное меню &
	\begin{minipage}[t]{1\linewidth}
		\vspace{-1\baselineskip}
		\testnumber. \textbf{Создание лобби}.\newline
 		Предусловие: необходимо быть авторизованным.
 		\begin{enumerate}
 			\item В главном меню выбрать пункт <<Host game>>.
 			\item Задать имя лобби и выбрать набор вопросов для викторины.
    		\item Нажать кнопку <<Create lobby>>.
 		\end{enumerate}
 	\end{minipage} &
	\begin{minipage}[t]{1\linewidth}
		\vspace{-1\baselineskip}
		\begin{enumerate}
			\item Отображается страница создания лобби.
			\item Кнопка <<Create lobby>> становится активной.
   			\item Отображается игровое лобби, где пользователь является ведущим.
		\end{enumerate}
	\end{minipage} &
	\begin{minipage}[t]{1\linewidth}
		\vspace{-1\baselineskip}
		\centering Да
	\end{minipage} \\


	Главное меню &
	\begin{minipage}[t]{1\linewidth}
		\vspace{-1\baselineskip}
		\testnumber. \textbf{Подключение к лобби без пароля}.\newline
 		Предусловие: необходимо быть авторизованным.
 		\begin{enumerate}
			\item В главном меню выбрать пункт <<Join game>>.
			\item Выбрать лобби без пароля из списка доступных.
			\item Нажать кнопку <<Join>>.
 		\end{enumerate}
 	\end{minipage} &
	\begin{minipage}[t]{1\linewidth}
		\vspace{-1\baselineskip}
		\begin{enumerate}
			\item Отображается страница подключения к лобби.
			\item Кнопка <<Join lobby>> становится активной.
   			\item Отображается игровое лобби, где пользователь является игроком.
		\end{enumerate}
	\end{minipage} &
	\begin{minipage}[t]{1\linewidth}
		\vspace{-1\baselineskip}
		\centering Да
	\end{minipage} \\
	\hline

	Главное меню &
	\begin{minipage}[t]{1\linewidth}
		\vspace{-1\baselineskip}
		\testnumber. \textbf{Подключение к лобби с паролем}.\newline
 		Предусловие: необходимо быть авторизованным.
 		\begin{enumerate}
			\item В главном меню выбрать пункт <<Join game>>.
 			\item Выбрать лобби для подключения.
 			\item Нажать кнопку <<Join>>.
 			\item Ввести верный пароль.
 			\item Нажать кнопку <<Continue>>.
 		\end{enumerate}
 	\end{minipage} &
	\begin{minipage}[t]{1\linewidth}
		\vspace{-1\baselineskip}
		\begin{enumerate}
			\item Отображается страница подключения к лобби.
			\item Кнопка <<Join lobby>> становиться активной.
   			\item Отображается окно ввода пароля.
      		\item Кнопка <<Continue>> становится активной.
   			\item Отображается игровое лобби, где пользователь является игроком.
		\end{enumerate}
	\end{minipage} &
	\begin{minipage}[t]{1\linewidth}
		\vspace{-1\baselineskip}
		\centering Да
	\end{minipage} \\


	Главное меню &
	\begin{minipage}[t]{1\linewidth}
		\vspace{-1\baselineskip}
		\testnumber. \textbf{Смена авторизационных данных}.\newline
 		Предусловие: необходимо быть авторизованным.
 		\begin{enumerate}
			\item В главном меню выбрать пункт <<Change credentials>>.
 			\item Ввести имя пользователя.
 			\item Нажать кнопку <<Use credentials>>.
 		\end{enumerate}
 	\end{minipage} &
	\begin{minipage}[t]{1\linewidth}
		\vspace{-1\baselineskip}
		\begin{enumerate}
			\item Отображается страница авторизации.
			\item Кнопка использования данных становится активной.
   			\item Отображается главное меню. Над главным меню отображается выбранный аватар и имя пользователя.
		\end{enumerate}
	\end{minipage} \newline &
	\begin{minipage}[t]{1\linewidth}
		\vspace{-1\baselineskip}
		\centering Да
	\end{minipage} \\
	\hline

	Игровое лобби &
	\begin{minipage}[t]{1\linewidth}
		\vspace{-1\baselineskip}
		\testnumber. \textbf{Начало игры}.\newline
 		Предусловие: необходимо быть хостом лобби.
 		\begin{enumerate}
			\item Ждать подключения всех игроков.
 			\item Нажать кнопку <<Start game>>.
 		\end{enumerate}
 	\end{minipage} &
	\begin{minipage}[t]{1\linewidth}
		\vspace{-1\baselineskip}
		\begin{enumerate}
			\item Отображается страница игрового лобби со списком подключенных игроков.
			\item Отображается список вопросов первого раунда.
   			\item Хост получает право выбора игрока для первого хода.
		\end{enumerate}
	\end{minipage} &
	\begin{minipage}[t]{1\linewidth}
		\vspace{-1\baselineskip}
		\centering Да
	\end{minipage} \\
	\hline

	Игровое лобби &
	\begin{minipage}[t]{1\linewidth}
		\vspace{-1\baselineskip}
		\testnumber. \textbf{Выбор игрока для первого хода}.\newline
 		Предусловие: необходимо начать игру и быть хостом лобби.
 		\begin{enumerate}
			\item Нажать на аватар любого из подключенных игроков.
 		\end{enumerate}
 	\end{minipage} &
	\begin{minipage}[t]{1\linewidth}
		\vspace{-1\baselineskip}
		\begin{enumerate}
			\item Выбранный игрок получает право выбора вопроса.
		\end{enumerate}
	\end{minipage} &
	\begin{minipage}[t]{1\linewidth}
		\vspace{-1\baselineskip}
		\centering Да
	\end{minipage} \\
	

	Игровое лобби &
	\begin{minipage}[t]{1\linewidth}
		\vspace{-1\baselineskip}
		\testnumber. \textbf{Выбор вопроса}.\newline
 		Предусловие: иметь право выбора вопроса.
 		\begin{enumerate}
			\item Выбрать неразыгранный вопрос из списка вопросов текущего раунда.
 		\end{enumerate}
 	\end{minipage} &
	\begin{minipage}[t]{1\linewidth}
		\vspace{-1\baselineskip}
		\begin{enumerate}
			\item Отображается содержимое вопроса.
   			\item У игроков появляется кнопка <<Answer question>>
		\end{enumerate}
	\end{minipage} &
	\begin{minipage}[t]{1\linewidth}
		\vspace{-1\baselineskip}
		\centering Да
	\end{minipage} \\
	\hline

	Игровое лобби &
	\begin{minipage}[t]{1\linewidth}
		\vspace{-1\baselineskip}
		\testnumber. \textbf{Ответ на вопрос игроком}.\newline
 		Предусловие: в данный момент отображается содержимое вопроса.
 		\begin{enumerate}
			\item Нажать кнопку <<Answer question>>.
 			\item Дать ответ организатору в устной форме.
 			\item Дождаться оценки ответа организатором.
 		\end{enumerate}
 	\end{minipage} &
	\begin{minipage}[t]{1\linewidth}
		\vspace{-1\baselineskip}
		\begin{enumerate}
			\item У организатора появляются кнопки оценки правильности ответа.
   			\item Изменение игрового баланса в соответствии с правильностью ответа.
		\end{enumerate}
	\end{minipage} &
	\begin{minipage}[t]{1\linewidth}
		\vspace{-1\baselineskip}
		\centering Да
	\end{minipage} \\
	\hline

	Игровое лобби &
	\begin{minipage}[t]{1\linewidth}
		\vspace{-1\baselineskip}
		\testnumber. \textbf{Положительная оценка правильности ответа организатором}.\newline
 		Предусловие: игрок дает ответ на вопрос.
 		\begin{enumerate}
			\item Нажать кнопку <<Correct answer>>.
 		\end{enumerate}
 	\end{minipage} &
	\begin{minipage}[t]{1\linewidth}
		\vspace{-1\baselineskip}
		\begin{enumerate}
			\item Игроку начисляются очки, равные стоимости вопроса.
   			\item Изменение игрового баланса в соответствии с правильностью ответа.
		\end{enumerate}
	\end{minipage} &
	\begin{minipage}[t]{1\linewidth}
		\vspace{-1\baselineskip}
		\centering Да
	\end{minipage} \\
	\hline
	

	Игровое лобби &
	\begin{minipage}[t]{1\linewidth}
		\vspace{-1\baselineskip}
		\testnumber. \textbf{Отображение победителей}.\newline
 		Предусловие: остался один неразыгранный вопрос.
 		\begin{enumerate}
			\item Разыграть последний вопрос.
 		\end{enumerate}
 	\end{minipage} &
	\begin{minipage}[t]{1\linewidth}
		\vspace{-1\baselineskip}
		\begin{enumerate}
			\item Отображается надпись Quiz is over.
   			\item Над игроками победителями отображается надпись Winner.
		\end{enumerate}
	\end{minipage} \newline &
	\begin{minipage}[t]{1\linewidth}
		\vspace{-1\baselineskip}
		\centering Да
	\end{minipage} \\
	\hline

	\end{longtable}


	% Зачем: зануляем счетчик для следующей таблицы.
	\setcounter{testnumber}{0}
	
	\begin{longtable}[l]{|>{\centering}m{0.15\textwidth}
					  |p{0.8\textwidth}
					  |p{0.34\textwidth}
					  |>{\centering\arraybackslash}m{0.16\textwidth}|} 
	\caption{Тестовые случаи негативного тестирования}
	\label{table:testing:negative}\\

	\hline
	\centering Модуль (экран) & \centering Описание тестового случая & \centeringОжидаемые результаты & \centering\arraybackslash Тестовый случай пройден? \endfirsthead

	\caption*{Продолжение таблицы \ref{table:testing:negative}}\\\hline
	\centering 1 & \centering 2 & \centering 3 & \centering\arraybackslash 4 \\\hline \endhead

	\hline
	\centering 1 & \centering 2 & \centering 3 & \centering\arraybackslash 4 \\
	\hline

	Автори-зация &
	\begin{minipage}[t]{1\linewidth}
		\vspace{-1\baselineskip}
		\testnumber. \textbf{Авторизация без данных}.\newline
 		Предусловие: необходимо находится на странице авторизации.
 		\begin{enumerate}
			\item Ничего не вводить в поле имени пользователя.
 		\end{enumerate}
 	\end{minipage} &
	\begin{minipage}[t]{1\linewidth}
		\vspace{-1\baselineskip}
		\begin{enumerate}
			\item Кнопка использования данных неактивна.
   			\item Используется аватар по умолчанию.
		\end{enumerate}
	\end{minipage} &
	\begin{minipage}[t]{1\linewidth}
		\vspace{-1\baselineskip}
		\centering Да
	\end{minipage} \\
	\hline

	Создание лобби &
	\begin{minipage}[t]{1\linewidth}
		\vspace{-1\baselineskip}
		\testnumber. \textbf{Создание лобби с неправильными данными}.\newline
 		Предусловие: необходимо находится на странице создания лобби.
 		\begin{enumerate}
			\item Ничего не вводить в поле названия лобби.
   			\item Не выбирать пакет вопросов.
 		\end{enumerate}
 	\end{minipage} &
	\begin{minipage}[t]{1\linewidth}
		\vspace{-1\baselineskip}
		\begin{enumerate}
			\item Кнопка создания лобби неактивна.
   			\item Около поля ввода названия лобби отображается валидационная ошибка "lobby name is empty".
   			\item В поле пути к пакету вопросов отображется "quiz pack is not set"
		\end{enumerate}
	\end{minipage} &
	\begin{minipage}[t]{1\linewidth}
		\vspace{-1\baselineskip}
		\centering Да
	\end{minipage} \\


	Подключе-ние к лобби &
	\begin{minipage}[t]{1\linewidth}
		\vspace{-1.5\baselineskip}
		\testnumber. \textbf{Подключение к лобби с неправильным паролем}.\newline
 		Предусловие: необходимо находится на странице подключения лобби.
 		\begin{enumerate}
			\item В главном меню выбрать пункт <<Join game>>.
 			\item Выбрать лобби для подключения из списка доступных.
 			\item Нажать кнопку <<Join>>.
 			\item Ввести неверный пароль.
 			\item Нажать кнопку <<Continue>>.
 		\end{enumerate}
 	\end{minipage} &
	\begin{minipage}[t]{1\linewidth}
		\vspace{-1.5\baselineskip}
		\begin{enumerate}
			\item Отображается страница подключения к лобби.
			\item Кнопка <<Join lobby>> становится активной.
   			\item Отображается окно для ввода пароля.
      		\item Кнопка <<Continue>> становится активной.
   			\item Отображется валидационная ошибка "password is invalid".
		\end{enumerate}
	\end{minipage} &
	\begin{minipage}[t]{1\linewidth}
		\vspace{-1.5\baselineskip}
		\centering Да
	\end{minipage} \\
	\hline

	\end{longtable}
\end{landscape}

% Зачем: возвращаем нумерацию перечислений как надо по стандарту.
\renewcommand{\labelenumi}{\asbuk{enumi})}
\renewcommand{\labelenumii}{\arabic{enumii})}