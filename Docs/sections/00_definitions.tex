\sectioncentered*{Определения и сокращения}
\label{sec:definitions}

В настоящей пояснительной записке применяются следующие определения и сокращения.
\\

\emph{Спецификация} -- документ, который желательно полно, точно и верифицируемо определяет требования, дизайн, поведение или другие характеристики 
компонента или системы, и, часто, инструкции для контроля выполнения этих требований.

\emph{Десктоп-приложение} -- полноценное приложение, работа которого не зависит от других приложений. 
Зачастую требует предварительной установки и (или) настройки.

\emph{Кроссплатформенность} -- способность программного обеспечения работать на нескольких аппартных платформах и (или) операционной системе.

\emph{Нативное программное средство} -- программное средство, специфичное для какой-либо конкретной платформы и (или) операционной системы.

\emph{Байт-код} -- стандартное промежуточное представление, в которое может быть переведена программа автоматическими средствами, для дальнейшей интерпретации другой программой.

\emph{Проприетарное программное обеспечение} -- программное обеспечение, являющееся частной собственностью авторов или правообладателей и не удовлетворяющее
критериям свободного ПО: свобода использования программного обеспечения в любых целях, модификация и адаптация исходного кода программы под свои нужны, свобода дистрибуции,
свобода улучшения исходного кода и публиция улучшений.
\\

ПС -- программное средство.

ПО -- программное обеспечение.

БД -- база данных.

XML -- extensible markup language (расширяемый язык разметки).

XAML -- extensible application markup language (расширяемый язык разметки приложений).

ЯП -- язык программирования.

ООП -- объектно-ориентированное программирование.

IL -- intermidiate language (промежуточный язык).

HTML -- hypertext markup language (язык разметки гипертекста).

CSS -- cascade stylesheet (каскадная страница стилей).

P2P -- peer-to-peer (от пользователя к пользователю)

LINQ -- language-integrated query (запрос, интегрированный в язык програмирования).

JSON -- javascript object notation (нотация объектов языка JavaScript).

CLR -- common language runtime (общая среда выполнения).

API -- application programming interface (программный интерфейс приложения).

UI -- user interface (пользовательский интерфейс).

ТЭО -- технико-экономическое обоснование.
