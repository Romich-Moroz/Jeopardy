\sectioncentered*{Введение}
\addcontentsline{toc}{section}{Введение}
\label{sec:introduction}

Развлечения всегда были и будут частью жизни любого человека. Самой первой игрой настольной игрой считается Сенет -- напоминающая современные шашки
и появившаяся за четыре тысячи лет до нашей еры в Египте. По мере развития технологий каждой эре были свойственны свои знаковые игры и развлечения.
Интересной особенностью являлось то, что в одну и ту же игру разные народы и культуры могли играть по разному, что порождало свои наборы правил, отличные от
оригинала и способствовало дальнейшему развитию игр.

Человеку близок соревновательный аспект игр: в древние времена сильнейший был самым успешныи в обществе. Благодаря таким развлечениям человек может доказать себе
и другим, что он достойный соперник и превосходит оппонентов без необходимости применения грубой силы. Множество настольных игр позволяют проверить такие качества, как:
терпение, ум, стратегическое мышление, память, коммуникативность и так далее. 

С развитием вычислительных машин появились и принципиально новые виды развлечений, позволяющие испытать себя в новых условиях без ущерба для здоровья. Например, шутеры,
появившиеся в начале девяностых годов, которые проверяют скорость реакции и координации -- навыки достаточно сложно проверяемые в настольных играх. В то время основным соперником
человека был искусственный интеллект под управлением компьютера, что ограничивало сложность соревнования навыками программистов, писавших искусственный интеллект.

С началом повсеместного распространения интернета закономерным развитием индустрии развлечений стало появление игр, расчитанных на нескольких игроков. Интерес таких игр повышался
многократно, по сравнению с играми, где был искусственный интеллект, так как человек по своей природе может быть достаточно непредсказуем, что повышает вариантивность игровых ситуаций,
к которым игрок должен адаптироваться. 

В настоящее время в мире существует множество жанров видеоигр, в которых участвует не два, не пять человек, а тридцать, шестьдесят или даже сотня человек одновременно. При этом 
общее количество игроков игре исчисляется сотнями тысяч. Большинство таких игр упирается на соревновательный аспект, например: сражение команда на команду или выявление 
лучшего среди всех. При этом порог входа в такие игры как правило не высок, но уровень знаний, необходимых для достижения лучших результов достаточно высокий, 
чтобы игроки оставались в игре на долгое время.

Целью настоящего дипломного проекта является разработка програм-много средства, которое позволяет участникам соревнования определить самого лучшего знатока предметной области
в игровой форме на основе формата викторины. Причем в качестве вопросов могут выступать как текст и картинки, так и музыка с видео. В результате проведения викторины все игроки
могут узнать свои и чужие результаты по количеству набранных баллов.

Программное средство должно помочь в решении следующих задач: проведение викторин с большим количеством игроков, посчет игровых очков для определения победителя, поддержка должного уровня
разнообразия самого процесса, за счет вопросов разных видов и типов, помощь в проведении для игроков из разных точек земного шара, за счет использования интернет-соединения.

В пояснительной записке к дипломному проекту излагаются детали поэтапной разработки программного средства проведения многопользовательских викторин.
В первом разделе приведены результаты анализа литературных источников по теме дипломного проекта, рассмотрены особенности существующих систем-аналого, выдвинуты требования
к проектируемому программному средству. Во втором разделе приведено описание функциональности проектируемого программного средства, представлена спецификация функциональных требований.
В третьем разделе приведены детали проектирования и конструирования программного средства. Результатом этапа конструирования является функционирующее программное средство. В четвертом
разделе представлены доказательство того, что спроетированное программно средство работает в соответствии с выдвинутыми требованиями спецификации. В пятом разделе приведены сведения
по развертыванию и запуску ПС, указаны требуемые аппаратные и программные средства. Обоснование целесообразности создания программного средства с технико-экономической точки зрения
приведено в шестом разделе. Итоги проектирования, конструирования программного средста, а также соответствующие выводы приведены в заключении.