\subsection{Требования к проектируемому программному средству}
\label{sec:analysis:requirements}

По результатам изучения предметной области, анализа литературных источников и обзора существующих систем-аналогов сформулируем требования к проектируемому программному средству.

\subsubsection{} Назначение проекта
\label{sec:analysis:requirements:designation}

Назначением проекта является разработка программного средства, позволяющего автоматизировать процессы проведения викторин на произвольные темы.

\subsubsection{} Основные функции
\label{sec:analysis:requirements:functions}

Программное средство должно поддерживать следующие основные функции:

\begin{itemize}
	\item создание и редактирование пакетов с вопросами, которые будут использоваться во время проведения викторин;
	\item поддержка нескольких форматов вопросов (текст, звук, видео и т. д.);
	\item поддержка нескольких видов вопросов (обычный, с секретом, со ставкой и т. д.);
	\item анонимная авторизация с помощью локально созданного аккаунта;
	\item создание игрового лобби, как организатор викторины;
	\item настройка игрового лобби для проведения викторины (выбор пакета вопросов, пароль и т. д.);
	\item присоединение к игровому лобби в качестве игрока;
	\item управление игровым процессом для ведущего;
	\item автоматическое определение победителя после ответов на все вопросы.
\end{itemize} 

\subsubsection{} Требования к входным данным
\label{sec:analysis:requirements:input}

Входные данные для программного средства должны представлять из себя вводимый с клавиатуры текст или выбор доступных опций на пользовательском интерфейсе.
Должны быть реализованы проверки вводимых данных на корректность с отображением информации об ошибка в случае их некорректности.

\subsubsection{} Требования к выходным данным
\label{sec:analysis:requirements:output}

Выходные данные программного средства должны быть представлены посредством отображения информации с помощью различных элементов пользовательского интерфейса.

\subsubsection{} Требования к временным характеристикам
\label{sec:analysis:requirements:time}

Производительность программно-аппаратного комплекса должна \linebreak обеспечивать следующие временные характеристики: время реакции на запрос пользователя не должно
превышать одной секунды при минимальной скорости соединения 1 МБит/c. Допускается невыполнение данного требования в случае, когда невозможно обеспечить заявленную
пропускную способность интернет-канала по внешним причинам и не зависящим от пользователя обстоятельствам.

\subsubsection{} Требования к надежности
\label{sec:analysis:requirements:reliability}

Надежное функционирование программы должно быть обеспечено выполнением следующих организационно-технических мероприятий:

\begin{itemize}
	\item организация бесперебойного питания;
	\item выполнение требований ГОСТ 31078-2002 «Защита информации. Испытания программных средств на наличие компьютерных вирусов»;
	\item выполнение рекомендаций Министерства труда и социальной защиты РБ, изложенных в Постановлении от 23 марта 2011 г. «Об утверждении норм времени на работы по обслуживанию персональных электронно-вычислительных машин, организационной техники и офисного оборудования»;
	\item необходимым уровнем квалификации пользователей.
\end{itemize} 

Время восстановления после отказа, вызванного сбоем электропитания технических средств (иными внешними факторами), 
нефатальным сбоем операционной системы, не должно превышать времени, необходимого на перезагрузку операционной системы и запуск программы, 
при условии соблюдения условий эксплуатации технических и программных средств. Время восстановления после отказа, вызванного неисправностью технических средств,
фатальным сбоем операционной системы, не должно превышать времени, требуемого на устранение неисправностей технических средств и переустановки программных средств.

Отказы программы возможны вследствие некорректных действий \linebreak пользователя при взаимодействии с операционной системой. 
Во избежание возникновения отказов программы по указанной выше причине следует обеспечить работу конечного пользователя без предоставления ему административных привилегий.

\subsubsection{} Требования к аппаратному обеспечению серверной части
\label{sec:analysis:requirements:hardware_server}

ЭВМ, на которой должна функционировать серверная часть програм-много средства, должна обладать следующими минимальными характеристиками:

\begin{itemize}
	\item процессор Intel Core i7 с тактовой частотой 4 ГГц;
	\item жесткий диск объемом 100 Гб;
	\item оперативная память 16 Гб;
	\item сетевая карта Ethernet 100 МБит/с.
\end{itemize} 

Также для функционирования серверной части требуется Docker, который является кроссплатформенным программные средством, вследствие чего вопрос о целевой операционной системе не
рассматривается. Кроме того, процедуры установки и настройки данного программного средства выходят за рамки данного проекта и также не рассматриваются.

\subsubsection{}Требования к аппаратному обеспечению клиентской части
\label{sec:analysis:requirements:hardware_client}

Клиентская часть программного средства должна функционировать на ЭВМ со следующими минимальными характеристиками:
\begin{itemize}
	\item процессор Intel Core i7 с тактовой частотой 4 ГГц;
	\item оперативная память 4 Гб;
	\item сетевая карта Ethernet 10/100 МБит/с.
\end{itemize} 

Для корректной работы программного средства необходимы следующие программные компоненты:
\begin{itemize}
    \item операционная система Windows 10;
	\item .NET Framework 4.8;
	\item visual c++ redistributable.
\end{itemize} 

\subsubsection{}Выбор технологий программирования
\label{sec:analysis:requirements:language}

Язык программирования, на котором будет реализована система, имеет большое значение, так как он будет использоваться с начала конструирования программы и до самого конца.
Исследования показали, что выбор языка программирования несколькими способами влияет на производительность труда программистов и качество создаваемого ими кода. Если язык
хорошо знаком программистам, они работают более производительно. Данные, полученные при помощи модели оценки Cocomo II, показывают, что программисты, использующие язык, 
с которым они работали три года или более, примерно на 30\% более продуктивны, чем программисты, обладающие аналогичным опытом, но для которых язык является новым \cite{software_cost_estimation}.
В более раннем исследовании, проведенном в IBM, было обнаружено, что программисты, обладающие богатым опытом использования языка программирования, были более чем втрое 
производительнее программистов, имеющих минимальный опыт \cite{method_of_programming_measurement_and_estimation}.

Язык программирования С\#, указанный в задании на дипломное проектирование, является кроссплатформенным языком программирования и является частью платформы .NET. Данный ЯП представляет собой самостоятельный язык, 
который имеет C-образный синтаксис и семантику управляющих конструкций. Так как язык является строго типизированным, то основные ошибки можно отловить еще на этапе компиляции проекта.
Свою популярность он снискал за небольшое количество легаси-кода и большое количество синтаксического "сахара", который позволяет компактно описывать сложные конструкции.
Еще одной сильной стороной является технология LINQ, которая позволяет работать с любыми коллекциями, будь то база данных или локальный массив одинаково, причем в стандарте уже
реализованы все основные операции работы с данными, такие как добавление, обновление, удаление, фильтрация и так далее. Язык компилируется в IL и в дальнейшем выполняется в 
CLR, что обеспечивает кроссплатформенность и поддержку принципа write once -- run anywhere (напиши один раз -- запуская везде).
На данном языке разработано множество библиотек для работы в самых разных областях, например: настольные приложения, веб-приложения, машинное обучение и так далее. Таким
образом программист, знающий С\#, может писать под что угодно, лишь за исключением систем, для которых объем памяти строго ограничен, к таким относятся встраиваемые системы.

Среди достоинств платформы .NET можно выделить следующие:

\begin{itemize}
	\item обеспечение согласованной объектно-ориентированной среды программирования для локального сохранения и выполнения объектного кода, для
    локального выполнения кода, распределенного в Интернете, либо для удаленного выполнения;
	\item обеспечение среды выполнения кода, минимизирующей конфликты
    при развертывании программного обеспечения и управлении версиями;
	\item обеспечение среды выполнения кода, гарантирующей безопасное выполнение кода, включая код, созданный неизвестным или не полностью доверенным сторонним изготовителем;
	\item обеспечение среды выполнения кода, исключающей проблемы с производительностью сред выполнения сценариев или интерпретируемого кода;
	\item обеспечение единых принципов работы разработчиков для разных типов приложений, таких как приложения Windows и веб-приложения;
	\item разработка взаимодействия на основе промышленных стандартов, которое обеспечит интеграцию кода платформы .NET с любым другим
    кодом.
\end{itemize} 

Исходя из достоинств данного языка программирования, можно сделать вывод, что он наиболее подходящий для решения проблем,
схожих с поднимаемыми в данной пояснительной записке. Именно поэтому С\# и выбран как основной язык программирования в задании к текущему дипломному проекту.
Он является простым, современным, объектно-ориенти-рованным, обеспечивающим безопасность типов языком программирования. Он соответствует международному стандарту Европейской ассоциации производителей компьютеров — стандарт ECMA-334, а
также стандарту Международной организации по стандартизации (International
Standards Organization, ISO) и Международной электротехнической комиссии
— стандарт ISO/IEC 23270. Компилятор Microsoft С\# для .NET согласуется с обоими этими стандартами \cite{msdn_charp}.

Передовой средой программирования, которая поддерживает С\# явля-ется Microsoft Visual Studio, которая входит в линейку
продуктов компании Microsoft, включающих интегрированную среду разработки программного обеспечения и ряд других инструментальных средств. Она
включает в себя редактор исходного кода с поддержкой технологии IntelliSense
и возможностью рефакторинга кода. Встроенный отладчик может работать как
отладчик уровня исходного кода, так и как отладчик машинного уровня. Visual
Studio позволяет создавать и подключать сторонние дополнения для расширения функциональности практически на каждом уровне, включая добавление
поддержки систем контроля версий исходного кода, добавление новых наборов инструментов или инструментов для прочих аспектов процесса разработки
программного обеспечения. Именно поэтому она и выбрана в качестве основной среды программирования.

Для разработки сервиса-координатора, который будет работать на сервере будет также использоваться С\# и технология сокетов с написанием собственного протокола передачи данных.
Данный протокол является необходимым по той причине, что клиентам и серверу необходимо понимать формат пакетов, приходящих друг от друга. Разрабатываемый сервер представляет из себя
приложение-сервис, которое работает постоянно и отвечает на запросы клиентов, при этом оно хранит некое промежуточное состояние работы, которое может использовать
при обработке запросов от клиентов. В случае разработки проектируемого программного средства веб-сервис будет хранить список активных игровых лобби, которые будут запрашивать
клиенты при попытке подключится к организатору викторины, а также игровые состояния для каждого лобби.

Сформулированные требования позволят осуществить успешное проектирование и разработку программного средства.
