\newcommand{\byn}{\text{руб.}}
\newcommand{\hour}{\text{ч.}}
\newcommand{\people}{\text{чел.}}

\newcommand{\premiumPercentFactor}{\text{К}_\text{пр}}
\newcommand{\additionalSalaryFactor}{\text{Н}_\text{д}}
\newcommand{\socialTaxFactor}{\text{Н}_\text{соц}}
\newcommand{\otherExpensesFactor}{\text{Н}_\text{пз}}
\newcommand{\vatFactor}{\text{Н}_\text{дс}}
\newcommand{\profitFactor}{\text{Н}_\text{п}}

\newcommand{\hourWagePerPerson}{\text{З}_\text{ч.i}}
\newcommand{\monthlyWagePerPerson}{\text{З}_\text{м.i}}
\newcommand{\totalHoursPerPerson}{\text{t}_\text{i}}
\newcommand{\totalHoursPerGroup}{\text{T}_\text{i}}
\newcommand{\salaryPerGroup}{\text{З}_\text{i}}
\newcommand{\salaryPrimaryWithoutPremium}{\text{З}_\text{общ}}
\newcommand{\premiumForPrimarySalary}{\text{Р}_\text{п}}
\newcommand{\salaryPrimaryWithPremium}{\text{З}_\text{о}}
\newcommand{\salaryAdditional}{\text{З}_\text{д}}
\newcommand{\socialTax}{\text{P}_\text{соц}}
\newcommand{\otherExpenses}{\text{P}_\text{пз}}
\newcommand{\totalExpenses}{\text{З}_\text{р}}

\newcommand{\totalHoursBI}{\text{t}_\text{1}}
\newcommand{\totalHoursArch}{\text{t}_\text{2}}
\newcommand{\totalHoursEng}{\text{t}_\text{3}}
\newcommand{\totalHoursProg}{\text{t}_\text{4}}
\newcommand{\totalHoursQA}{\text{t}_\text{5}}
\newcommand{\totalHoursFront}{\text{t}_\text{6}}

\newcommand{\vatsymbol}{\text{НДС}}

\newcommand{\licenseCount}{\text{N}}
\newcommand{\productPrice}{\text{Ц}}
\newcommand{\profitDirty}{\text{П}}
\newcommand{\profitPure}{\text{П}_\text{ч}}
\newcommand{\profitRate}{\text{У}_\text{р}}
\FPeval{\hoursPerMonth}{168}

\FPeval{\premiumPercentFactorValue}{30}
\FPeval{\additionalSalaryFactorValue}{15}
\FPeval{\socialTaxFactorValue}{34.6}
\FPeval{\otherExpensesFactorValue}{25}
\FPeval{\totalDifferentSpecialties}{6}
\FPeval{\vatFactorValue}{20}
\FPeval{\profitFactorValue}{13}


\FPeval{\totalHoursPerPersonValueBI}{240}
\FPeval{\totalHoursPerPersonValueArch}{160}
\FPeval{\totalHoursPerPersonValueEng}{250}
\FPeval{\totalHoursPerPersonValueProg}{300}
\FPeval{\totalHoursPerPersonValueQA}{350}
\FPeval{\totalHoursPerPersonValueFront}{160}

\FPeval{\monthlyWageValueBI}{3250}
\FPeval{\monthlyWageValueArch}{11500}
\FPeval{\monthlyWageValueEng}{6000}
\FPeval{\monthlyWageValueProg}{3500}
\FPeval{\monthlyWageValueQA}{2350}
\FPeval{\monthlyWageValueFront}{2500}

\FPeval{\personCountBI}{1}
\FPeval{\personCountArch}{1}
\FPeval{\personCountEng}{1}
\FPeval{\personCountProg}{2}
\FPeval{\personCountQA}{3}
\FPeval{\personCountFront}{1}

\FPeval{\productPriceValue}{15}
\FPeval{\licenseCountValue}{15000}

\FPeval{\rawHourWagePerBA}{\monthlyWageValueBI / \hoursPerMonth}
\FPround{\roundHourWagePerBA}{\rawHourWagePerBA}{2}
\newcommand{\hourWagePerBA}{\num{\roundHourWagePerBA}}
\FPeval{\rawHourWagePerArch}{\monthlyWageValueArch / \hoursPerMonth}
\FPround{\roundHourWagePerArch}{\rawHourWagePerArch}{2}
\newcommand{\hourWagePerArch}{\num{\roundHourWagePerArch}}
\FPeval{\rawHourWagePerEng}{\monthlyWageValueEng / \hoursPerMonth}
\FPround{\roundHourWagePerEng}{\rawHourWagePerEng}{2}
\newcommand{\hourWagePerEng}{\num{\roundHourWagePerEng}}
\FPeval{\rawHourWagePerProg}{\monthlyWageValueProg / \hoursPerMonth}
\FPround{\roundHourWagePerProg}{\rawHourWagePerProg}{2}
\newcommand{\hourWagePerProg}{\num{\roundHourWagePerProg}}
\FPeval{\rawHourWagePerQA}{\monthlyWageValueQA / \hoursPerMonth}
\FPround{\roundHourWagePerQA}{\rawHourWagePerQA}{2}
\newcommand{\hourWagePerQA}{\num{\roundHourWagePerQA}}
\FPeval{\rawHourWagePerFront}{\monthlyWageValueFront / \hoursPerMonth}
\FPround{\roundHourWagePerFront}{\rawHourWagePerFront}{2}
\newcommand{\hourWagePerFront}{\num{\roundHourWagePerFront}}

\FPeval{\rawTotalHoursBA}{\totalHoursPerPersonValueBI * \personCountBI}
\FPround{\roundTotalHoursBA}{\rawTotalHoursBA}{0}
\newcommand{\totalHoursValueBA}{\num{\roundTotalHoursBA}}
\FPeval{\rawTotalHoursArch}{\totalHoursPerPersonValueArch * \personCountArch}
\FPround{\roundTotalHoursArch}{\rawTotalHoursArch}{0}
\newcommand{\totalHoursValueArch}{\num{\roundTotalHoursArch}}
\FPeval{\rawTotalHoursEng}{\totalHoursPerPersonValueEng * \personCountEng}
\FPround{\roundTotalHoursEng}{\rawTotalHoursEng}{0}
\newcommand{\totalHoursValueEng}{\num{\roundTotalHoursEng}}
\FPeval{\rawTotalHoursProg}{\totalHoursPerPersonValueProg * \personCountProg}
\FPround{\roundTotalHoursProg}{\rawTotalHoursProg}{0}
\newcommand{\totalHoursValueProg}{\num{\roundTotalHoursProg}}
\FPeval{\rawTotalHoursQA}{\totalHoursPerPersonValueQA * \personCountQA}
\FPround{\roundTotalHoursQA}{\rawTotalHoursQA}{0}
\newcommand{\totalHoursValueQA}{\num{\roundTotalHoursQA}}
\FPeval{\rawTotalHoursFront}{\totalHoursPerPersonValueFront * \personCountFront}
\FPround{\roundTotalHoursFront}{\rawTotalHoursFront}{0}
\newcommand{\totalHoursValueFront}{\num{\roundTotalHoursFront}}

\FPeval{\rawSalaryBA}{\rawTotalHoursBA * \rawHourWagePerBA}
\FPround{\roundSalaryBA}{\rawSalaryBA}{2}
\newcommand{\salaryValueBA}{\num{\roundSalaryBA}}
\FPeval{\rawSalaryArch}{\rawTotalHoursArch * \rawHourWagePerArch}
\FPround{\roundSalaryArch}{\rawSalaryArch}{2}
\newcommand{\salaryValueArch}{\num{\roundSalaryArch}}
\FPeval{\rawSalaryEng}{\rawTotalHoursEng * \rawHourWagePerEng}
\FPround{\roundSalaryEng}{\rawSalaryEng}{2}
\newcommand{\salaryValueEng}{\num{\roundSalaryEng}}
\FPeval{\rawSalaryProg}{\rawTotalHoursProg * \rawHourWagePerProg}
\FPround{\roundSalaryProg}{\rawSalaryProg}{2}
\newcommand{\salaryValueProg}{\num{\roundSalaryProg}}
\FPeval{\rawSalaryQA}{\rawTotalHoursQA * \rawHourWagePerQA}
\FPround{\roundSalaryQA}{\rawSalaryQA}{2}
\newcommand{\salaryValueQA}{\num{\roundSalaryQA}}
\FPeval{\rawSalaryFront}{\rawTotalHoursFront * \rawHourWagePerFront}
\FPround{\roundSalaryFront}{\rawSalaryFront}{2}
\newcommand{\salaryValueFront}{\num{\roundSalaryFront}}

\FPeval{\rawSalaryValue}{\rawSalaryBA + \rawSalaryArch + \rawSalaryEng + \rawSalaryProg + \rawSalaryQA + \rawSalaryFront}
\FPround{\roundSalaryValue}{\rawSalaryValue}{2}
\newcommand{\salaryPrimaryWithoutPremiumValue}{\num{\roundSalaryValue}}
\FPeval{\rawSalaryPremiumValue}{\rawSalaryValue * \premiumPercentFactorValue/100}
\FPround{\roundSalaryPremiumValue}{\rawSalaryPremiumValue}{2}
\newcommand{\premiumForSalaryPrimaryValue}{\num{\roundSalaryPremiumValue}}
\FPeval{\rawSalaryTotalValue}{\rawSalaryValue + \rawSalaryPremiumValue}
\FPround{\roundSalaryTotalValue}{\rawSalaryTotalValue}{2}
\newcommand{\salaryPrimaryWithPremiumValue}{\num{\roundSalaryTotalValue}}

\FPeval{\rawSalaryAdditionalValue}{\rawSalaryTotalValue * \additionalSalaryFactorValue/100}
\FPround{\roundSalaryAdditionalValue}{\rawSalaryAdditionalValue}{2}
\newcommand{\salaryAdditionalValue}{\num{\roundSalaryAdditionalValue}}
\FPeval{\rawSocialTaxValue}{(\rawSalaryTotalValue + \rawSalaryAdditionalValue) * \socialTaxFactorValue / 100}
\FPround{\roundSocialTaxValue}{\rawSocialTaxValue}{2}
\newcommand{\socialTaxValue}{\num{\roundSocialTaxValue}}
\FPeval{\rawOtherExpensesValue}{\rawSalaryTotalValue * \otherExpensesFactorValue / 100}
\FPround{\roundOtherExpensesValue}{\rawOtherExpensesValue}{2}
\newcommand{\otherExpensesValue}{\num{\roundOtherExpensesValue}}
\FPeval{\rawTotalExpensesValue}{\rawSalaryTotalValue + \rawSalaryAdditionalValue + \rawSocialTaxValue + \rawOtherExpensesValue}
\FPround{\roundTotalExpensesValue}{\rawTotalExpensesValue}{2}
\newcommand{\totalExpensesValue}{\num{\roundTotalExpensesValue}}

\FPeval{\rawVatValue}{(\productPriceValue * \licenseCountValue * \vatFactorValue/100) / (1 + \vatFactorValue/100)}
\FPround{\roundVatValue}{\rawVatValue}{2}
\newcommand{\totalVatValue}{\num{\roundVatValue}}
\FPeval{\rawProfitDirtyValue}{\productPriceValue * \licenseCountValue - \rawVatValue - \rawTotalExpensesValue}
\FPround{\roundProfitDirtyValue}{\rawProfitDirtyValue}{2}
\newcommand{\profitDirtyValue}{\num{\roundProfitDirtyValue}}
\FPeval{\rawProfitPureValue}{(\rawProfitDirtyValue * (1 - \profitFactorValue / 100))}
\FPround{\roundProfitPureValue}{\rawProfitPureValue}{2}
\newcommand{\profitPureValue}{\num{\roundProfitPureValue}}
\FPeval{\rawProfitRateValue}{(\rawProfitPureValue / \rawTotalExpensesValue * 100)}
\FPround{\roundProfitRateValue}{\rawProfitRateValue}{2}
\newcommand{\profitRateValue}{\num{\roundProfitRateValue}}

\section{Технико-экономическое обоснование разработки программного средства проведения многопользовательских викторин}
\label{sec:economics}

\subsection{Характеристика программного средства}
\label{sec:economics:description}

Разработка программ связана со значительными затратами ресурсов и всевозможными рисками. Перед разработкой проводится соответствующее 
технико-экономическое обоснование, позволяющее предварительно оценить затраты и выгоду от инвестиций в разработку программного средства. Это обоснование и описывается в данном разделе.

\subsection{Описание функций, назначения и потенциальных пользователей ПО}
\label{sec:economics:functions_and_target_audience}

Разработанное программное обеспечение служит для автоматизации проведения викторин на произвольные темы в игровой форме. Одним из видов использования является рекреационный, то есть получение
удовольствия от игрового процесса. В то же время некоторые пользователи могут использовать формат викторин для повышения собственной эрудиции по наиболее интересным темам. Также допустим
вариант проверки знаний по уже изученным темам в незамысловатой форме. Программное средство берет на себя большую часть организации хода проведения викторины и оставляет за ведущим лишь
возможность принятия ключевых решений, что значительно ускоряет сам процесс игры и снижает нагрузку на организатора, что особенно заметно, если присутствует много игроков. Среди главных функций
можно выделить несколько видов вопросов: стандартный, секрет, со ставкой и так далее, несколько форматов вопросов: текст, звук, изображение, видео, а также поддержку множества пользователей
с работой по сети Интернет. Посчет результатов викторины и выявление победителя выполняется автоматически.

Основными потенциальными покупателями данного программного средства являются люди, желающие весело и пользой провести время с друзьями, причем местоположение друзей не имеет значения, так как работа
приложения происходит по сети. Вторичными потенциальными покупателями являются учебные заведения, которые могу проводить образовательные мероприятия для детей или студентов в игровой форме.
После анализа рынка было выяснено, что данный формат досуга пользуется достаточно большим спросом и существующие аналоги не способны удовлетворить его в полном объеме: в час пик качество услуг аналогов
значительно снижается, вплоть до невозможности использования. Отчасти это обусловлено текущим переходом на удаленные виды работы и как следствие уменьшением количества встреч вживую. Поэтому данное 
программное средство может уверенно занять часть рынка развлечений. Сильной стороной данного формата развлечений является то, что он подходит для людей всех возрастов, что значительно
повышает охват целевой аудитории.

В результате использования разработанного программного обеспечения пользователи удовлетворят свои рекреационные потребности, повысят свой уровень знаний, а также определят, кто 
в их окружении самый эрудированный.

\subsection{Расчет затрат на разработку ПО}
\label{sec:economics:expenses}

Расчет затрат на разработку программного обеспечения включает в себя:
\begin{itemize}
    \item затраты на основную заработную плату разработчиков;
    \item затраты на дополнительную заработную плату разработчиков;
    \item отчисления на социальные нужды;
    \item прочие затраты.
\end{itemize}

Исходные данные сотрудников, которые будут использоваться при расчете сметы затрат, представлены в таблице~\ref{table:economics:expenses:initial_data_employees}.
Для расчетов используются коэффициенты из таблицы~\ref{table:economics:expenses:initial_data_factors}

\begin{table}
\caption{Исходные данные сотрудников}
\label{table:economics:expenses:initial_data_employees}
\centering
    \begin{tabular}{{ 
    |>{\raggedright}m{0.6\textwidth} | 
        >{\centering}m{0.17\textwidth} | 
        >{\centering\arraybackslash}m{0.15\textwidth}|}}

        \hline
    {\begin{center} Наименование показателя \end{center}} & Единицы измерения &	Значение \\
    
    \hline
    Общее количество часов работы каждого бизнес-аналитика & $\hour$ &\totalHoursPerPersonValueBI \\

    \hline
    Общее количество часов работы каждого системного архитектора & $\hour$ &\totalHoursPerPersonValueArch \\

    \hline
    Общее количество часов работы каждого инженера-программиста & $\hour$ &\totalHoursPerPersonValueEng \\

    \hline
    Общее количество часов работы каждого программиста & $\hour$ &\totalHoursPerPersonValueProg \\

    \hline
    Общее количество часов работы каждого специалиста по тестированию программного обеспечения & $\hour$ &\totalHoursPerPersonValueQA \\

    \hline
    Общее количество часов работы каждого дизайнера & $\hour$ &\totalHoursPerPersonValueFront \\

    \hline
    Ежемесячная заработная плата бизнес-аналитиков & $\byn$ &\monthlyWageValueBI \\

    \hline
    Ежемесячная заработная плата системных архитекторов & $\byn$ &\monthlyWageValueArch \\

    \hline
    Ежемесячная заработная плата инженеров-программистов & $\byn$ &\monthlyWageValueEng \\

    \hline
    Ежемесячная заработная плата программистов & $\byn$ &\monthlyWageValueProg \\

    \hline
    Ежемесячная заработная плата специалистов по тестированию программного обеспечения & $\byn$ &\monthlyWageValueQA \\

    \hline
    Ежемесячная заработная плата дизайнеров & $\byn$ &\monthlyWageValueFront \\

    \hline
    Число сотрудников на должности бизнес-аналитика & $\people$ &\personCountBI \\

    \hline
    Число сотрудников на должности системного архитектора & $\people$ &\personCountArch \\

    \hline
    Число сотрудников на должности инженера-программиста & $\people$ &\personCountEng \\

    \hline
    Число сотрудников на должности программиста & $\people$ &\personCountProg \\

    \hline
    Число сотрудников на должности специалиста по тестированию программного обеспечения & $\people$ &\personCountQA \\

    \hline
    Число сотрудников на должности дизайнера & $\people$ &\personCountFront \\
    
    \hline
    \end{tabular}
\end{table}

\begin{table}
\caption{Коэффициенты для вычислений}
\label{table:economics:expenses:initial_data_factors}
\centering
    \begin{tabular}{{ 
    |>{\raggedright}m{0.6\textwidth} | 
        >{\centering}m{0.17\textwidth} | 
        >{\centering\arraybackslash}m{0.15\textwidth}|}}

        \hline
    {\begin{center} Наименование показателя \end{center}} & Условное обозначение &	Значение \\
    
    \hline
    Коэффициент, учитывающий процент премий & $\premiumPercentFactor$ &\premiumPercentFactorValue \% \\

    \hline
    Норматив дополнительной заработной платы & $\additionalSalaryFactor$ &\additionalSalaryFactorValue \% \\

    \hline
    Норматив отчислений от фонда оплаты труда & $\socialTaxFactor$ &\socialTaxFactorValue \% \\

    \hline
    Норматив прочих затрат & $\otherExpensesFactor$ &\otherExpensesFactorValue \% \\

    \hline
    Норматив налога на добавленную стоимость & $\vatFactor$ &\vatFactorValue \% \\

    \hline
    Налог на прибыль для IT компаний & $\profitFactor$ &\profitFactorValue \% \\

    \hline
    Цена за копию программного средства & $\productPrice$ &\productPriceValue \\

    \hline
    Предполагаемое количество проданных копий & $\licenseCount$ &\licenseCountValue \\

    \hline
    \end{tabular}
\end{table}
    
Для определения величины основной заработной платы участников команды сначала необходимо найти часовую заработную плату для каждого специалиста. Учитывая тот факт, что
время работы и заработная плата между всеми специалистами одного рода распределены одинаково, то для каждого специалиста в своей области часовую ставку можно высчитать один раз.
Формула и пример подсчета часовой ставки архитектора

\begin{equation}
    \hourWagePerPerson = \frac{\monthlyWagePerPerson}{168} = \frac{\monthlyWageValueArch}{168} = \hourWagePerArch,
\end{equation}
\begin{explanation}
где & $ \monthlyWagePerPerson $ & величина ежемесячной заработной платы специалиста в конкретной области;\\
& $ 168 $ & количество часов в одном рабочем месяце.
\end{explanation}

Трудоемкость работ позволяет оценить общее число часов, необходимое специалистам для полного выполнения работы. В случае, если специалистов в отделе работает несколько, то 
трудоемкость является суммой затрат времени каждого специалиста по отдельности. Формула подсчета трудоемкости и работ и пример подсчета для программиста

\begin{equation}
    \totalHoursPerGroup = \sum_{i=1}^{n} \totalHoursPerPerson = \sum_{i=1}^{\personCountProg} \totalHoursPerPersonValueProg = \totalHoursValueProg,
\end{equation}
\begin{explanation}
где & $ \totalHoursPerPerson $ & общее количество часов работы одного сотрудника конкретной специальности;\\
& $ n $ & количество сотрудников выбранной специальности.
\end{explanation}

Сумму основной заработной платы без учета премий можно найти умножив часовую ставку сотрудника отдела на общий объем требуемой работы отдела. Эта величина показывает, сколько денег уйдет
на выплату заработной платы всему отделу при условии, что премии еще не выплачивались. Пример расчета приведен для программистов.
Для расчета суммы заработной платы всех сотрудников одной специальности без учета премии используется формула

\begin{equation}
    \salaryPerGroup = \totalHoursPerGroup \cdot \hourWagePerPerson = \totalHoursValueProg \cdot \hourWagePerProg = \salaryValueProg,
\end{equation}
\begin{explanation}
где & $ \totalHoursPerPerson $ & общее количество часов работы одного сотрудника конкретной специальности;\\
& $ n $ & количество сотрудников выбранной специальности.
\end{explanation}

Сумма основной заработной платы без учета премии показывает сколько нужно выплатить всем отделам разработки. Расчет и формула имеют следующий вид

\begin{equation}
    \salaryPrimaryWithoutPremium = \sum_{i=1}^{n} \salaryPerGroup = \salaryValueBA + \salaryValueArch + ... + \salaryValueFront = \salaryPrimaryWithoutPremiumValue,
\end{equation}
\begin{explanation}
где & $ \salaryPerGroup $ & сумма основной заработной платы без учета премии конкретной специальности;\\
& $ n $ & количество различных специальностей в команде разработчиков.
\end{explanation}

Премиальные отчисления необходимы для стимулирования сотрудников и повышения их заинтересованности в выполнении работы. Премиальная ставка показывает процент от первичной заработной платы,
который выплачивается в качестве премии за хорошую работы. Величину премий для каждого отдела можно найти по формуле

\begin{equation}
    \premiumForPrimarySalary = \frac{\salaryPrimaryWithoutPremium \cdot \premiumPercentFactor}{100\%} = \frac{\salaryPrimaryWithoutPremiumValue \cdot \premiumPercentFactorValue\%}{100\%} = \premiumForSalaryPrimaryValue,
\end{equation}
\begin{explanation}
где & $ \salaryPrimaryWithoutPremium $ & общая сумма выплат основной заработной платы без учета премии;\\
& $ \premiumPercentFactor $ & коэффициент, учитывающий размер премии.
\end{explanation}

Итоговая основная заработная плата содержит информацию о затратах на поддержку работы специалистов с учетом премии. Расчет проводится по формуле
\begin{equation}
    \salaryPrimaryWithPremium = \salaryPrimaryWithoutPremium + \premiumForPrimarySalary = \salaryPrimaryWithoutPremiumValue + \premiumForSalaryPrimaryValue = \salaryPrimaryWithPremiumValue,
\end{equation}
\begin{explanation}
где & $ \salaryPrimaryWithoutPremium $ & общая сумма выплат основной заработной платы без учета премии;\\
& $ \premiumForPrimarySalary $ & общая сумма премий.
\end{explanation}

Результаты расчетов основной заработной платы приведены в таблице~\ref{table:economics:expenses:primary_wages}.
    
\begin{table}[!ht]
\caption{Расчет затрат на основную заработную плату разработчиков}
\label{table:economics:expenses:primary_wages}
\centering
    \begin{tabular}{| >{\raggedright}m{0.18\textwidth} 
        | >{\centering}m{0.24\textwidth} 
        | >{\centering}m{0.13\textwidth} 
        | >{\centering}m{0.10\textwidth} 
        | >{\centering\arraybackslash}m{0.10\textwidth}
        | >{\centering\arraybackslash}m{0.11\textwidth}|}

        \hline
    {\begin{center} Наименование должности разработчика \end{center}} & Виды выполняемой работы & Месячная заработная плата, р. & Часовая заработная плата р. & Трудо-емкость работ, ч. & Сумма, р. \\
    
    \hline
    Бизнес-аналитик & Сбор и анализ требований, изучение пожеланий пользователей &\monthlyWageValueBI &\hourWagePerBA &\totalHoursValueBA &\salaryValueBA \\

    \hline
    Системный архитектор & Проектирование архитектуры ПС & \monthlyWageValueArch &\hourWagePerArch &\totalHoursValueArch &\salaryValueArch \\

    \hline
    Инженер-программист & Разработка важнейших компонентов ПС & \monthlyWageValueEng &\hourWagePerEng &\totalHoursValueEng &\salaryValueEng \\

    \hline
    Программист & Разработка основных компонентов ПС & \monthlyWageValueProg &\hourWagePerProg &\totalHoursValueProg &\salaryValueProg \\

    \hline
    Специалист по тестированию программного обеспечения & Тестирование ПС &\monthlyWageValueQA &\hourWagePerQA &\totalHoursValueQA &\salaryValueQA \\

    \hline
    Дизайнер & Разработка пользовательского интерфейса & \monthlyWageValueFront &\hourWagePerFront &\totalHoursValueFront &\salaryValueFront \\

    \hline
    \multicolumn{5}{|l|}{Итого} &\salaryPrimaryWithoutPremiumValue \\
    
    \hline
    \multicolumn{5}{|l|}{Премия (30\%)} &\premiumForSalaryPrimaryValue \\

    \hline
    \multicolumn{5}{|l|}{Всего основная заработная плата} &\salaryPrimaryWithPremiumValue \\

    \hline
    \end{tabular}
\end{table}

В список расходов на разработку также входят затраты на дополнительную заработную плату разработиков. Сюда входят выплаты, предусмотренные законодательством о труде, расчеты
производятся по формуле

\begin{equation}
    \salaryAdditional = \frac{\salaryPrimaryWithPremium \cdot \additionalSalaryFactor}{100\%} = \frac{\salaryPrimaryWithPremiumValue \cdot \additionalSalaryFactorValue\%}{100\%} = \salaryAdditionalValue,
\end{equation}
\begin{explanation}
где & $ \salaryPrimaryWithPremium $ & сумма основной заработной платы с учетом премии;\\
& $ \premiumPercentFactor $ & норматив дополнительной заработной платы.
\end{explanation}

Нормы отсчислений на социальные нужны (в фонд социальной защиты населения и на обязательное страхование) устанавливаются государством и любой бизнес обязан их выплачивать.
В соответствии с действующими законодательными актами.

\begin{equation}
    \socialTaxFactor = \frac{(\salaryPrimaryWithPremium + \salaryAdditional) \cdot \socialTaxFactor}{100\%} = \frac{(\salaryPrimaryWithPremiumValue + \salaryAdditionalValue) \cdot \socialTaxFactorValue\%}{100\%} = \socialTaxValue,
\end{equation}
\begin{explanation}
где & $ \salaryPrimaryWithPremium $ & сумма основной заработной платы с учетом премии;\\
& $ \salaryAdditional $ & сумма дополнительной заработной платы;\\
& $ \socialTaxFactor $ & норматив отчислений от фонда оплаты труда.
\end{explanation}

Прочие затраты, включающие в себя лицензии на программное обеспечени, используемое в разработке можно рассчитать по формуле

\begin{equation}
    \otherExpenses = \frac{\salaryPrimaryWithPremium \cdot \otherExpensesFactor}{100\%} = \frac{\salaryPrimaryWithPremiumValue \cdot \otherExpensesFactorValue\%}{100\%} = \otherExpensesValue,
\end{equation}
\begin{explanation}
где & $ \salaryPrimaryWithPremium $ & сумма основной заработной платы с учетом премии;\\
& $ \otherExpensesFactor $ & норматив прочих затрат.
\end{explanation}

Результаты подсчетов сведены в таблицу~\ref{table:economics:expenses:final}.

\begin{table}[!h]
\caption{Затраты на разработку программного обеспечения}
\label{table:economics:expenses:final}
\centering
    \begin{tabular}{{ 
    |>{\raggedright}m{0.75\textwidth} | 
        >{\centering\arraybackslash}m{0.22\textwidth}|}}

        \hline
    {\begin{center} Наименование статьи затрат \end{center}} & Значение, \byn \\
    
    \hline
    Основная заработная плата разработчиков &\salaryPrimaryWithPremiumValue \\

    \hline
    Дополнительная заработная плата разработчиков &\salaryAdditionalValue \\

    \hline
    Отчисления на социальные нужды &\socialTaxValue \\

    \hline
    Прочие затраты &\otherExpensesValue \\

    \hline
    Общая сумма инвестиций в разработку &\totalExpensesValue \\

    \hline
    \end{tabular}
\end{table}

\subsection{Оценка результата от продажи ПО}
\label{sec:economics:evaluation}

Цена за копию продукта была выбрана на основе статистики по средним ценам игр категории "инди" (малобюджетные), собранной на самой популярной платформе для дистрибуции видеоигр Steam. 
Средняя цена в эквиваленте на белорусские рубли составляет около 15 рублей \cite{steam_avg_price}. При этом количество проданных копий в сегменте "инди" обычно варьируется от 5-10 тысяч 
до сотни тысяч, а в особо успешных случаях количество копий может достигать полумиллиона и больше. Для расчетов возьмем среднее значение проданных копий, равное \licenseCountValue \cite{steam_avg_sales}. 

Для расчета прибыли, которую получит разработчик программного обеспечения, сначала необходимо рассчитать налог на добавленную стоимость по формуле

\begin{equation}
    \vatsymbol = \frac{(\productPrice \cdot \licenseCount \cdot \vatFactor)}{100\% + \vatFactor} = \frac{(\productPriceValue \cdot \licenseCountValue \cdot \vatFactorValue\%)}{100\% + \vatFactorValue\%} = \totalVatValue,
\end{equation}
\begin{explanation}
где & $ \productPrice $ & цена за одну лицензионную копию программного средства;\\
& $ \licenseCount $ & количество проданных копий;\\
& $ \vatFactor $ & коэффициент подоходного налога (в Беларуси равен \vatFactorValue\%).
\end{explanation}

Имея значение налога на добавленную стоимость можно рассчитать прибыль, полученную разработчиком от реализации программного обеспечения без учета налога на прибыль

\begin{equation}
    \profitDirty = \productPrice \cdot \licenseCount - \vatsymbol - \totalExpenses = \productPriceValue \cdot \licenseCountValue - \totalVatValue - \totalExpensesValue = \profitDirtyValue,
\end{equation}
\begin{explanation}
где & $ \totalExpenses $ & затраты на разработку программного обеспечения;\\
& $ \productPrice $ & цена за одну лицензионную копию программного средства;\\
& $ \licenseCount $ & количество проданных копий;\\
& $ \vatFactor $ & коэффициент подоходного налога (в Беларуси равен \vatFactorValue\%).
\end{explanation}

Для IT-компаний в Беларуси введены налоговые льготы на прибыль, поэтому при расчете чистой прибыли используется льготная налоговая ставка равная \profitFactorValue\% \cite{president_lite_tax}

\begin{equation}
    \profitPure = \profitDirty \cdot (1 - \frac{\profitFactor\%}{100\%}) = \profitDirtyValue \cdot (1 - \frac{\profitFactorValue\%}{100\%}) = \profitPureValue,
\end{equation}
\begin{explanation}
где & $ \profitDirty $ & прибыль без учета налога на прибыль;\\
& $ \profitFactor $ & налоговая ставка на прибыль.
\end{explanation}

Рентабельность показывает насколько выгодны ивестиции в разработку программного обеспечения, для ее подсчета используется формула

\begin{equation}
    \profitRate = \frac{\profitPure}{\totalExpenses}\cdot100\% = \frac{\profitPureValue}{\totalExpensesValue}\cdot100\% = \profitRateValue\%,
\end{equation}
\begin{explanation}
где & $ \profitPure $ & прибыль с учетом налога на прибыль;\\
& $ \totalExpenses $ & затраты на разработку программного обеспечения.
\end{explanation}

Учитывая тот факт, что средняя ставка по депозиту равна 19\% \cite{bank_avg_deposit}, то можно сделать вывод, что инвестиции в разработку программного средства являются более выгодными, по
сравнению с депозитным вкладом, а полученные результаты расчетов свидетельствуют об эффективности разработки проектируемого программного средства.