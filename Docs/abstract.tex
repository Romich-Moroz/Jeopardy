\sectioncentered*{Реферат}
\thispagestyle{empty}

% Зачем: чтобы можно было вывести общее число страниц.
% Добавляется единица, поскольку последняя страница -- ведомость.
\FPeval{\totalpages}{round(\getpagerefnumber{LastPage} + 1, 0)}

% \begin{center}
% 	Пояснительная записка \num{\totalpages}~с., \num{\totfig{}}~рис., \num{\tottab{}}~табл., \num{\toteq{}}~формул, \num{\totref{}}~источников.
% 	\MakeUppercase{Программное средство, настольное приложение, викторины, развлечение, соревнование, множество игроков}
% \end{center}
\noindent
ПРОГРАММНОЕ СРЕДСТВО ПРОВЕДЕНИЯ МНОГОПОЛЬЗОВА-\linebreakТЕЛЬСКИХ ВИКТОРИН С ИСПОЛЬЗОВАНИЕМ ПЛАТФОРМЫ .NET: дипломный проект / Р. В. Мороз. -- Минск, БГУИР,
2022, -- п.з. -- \num{\totalpages}~с., чертежей (плакатов) -- 6 л. формата А1.
\newline

Цель настоящего дипломного проекта состоит в разработке программной системы, предназначенной для эффективной автоматизации проведения 
многопользовательских викторин для участников игрового процесса: игроков и ведущего. 

В процессе анализа предметной области были выделены основные аспекты процесса проведения викторин, которые в настоящее время практически не охвачены автоматизацией. 
Было проведено их исследование и моделирование. Кроме того, рассмотрены существующие средства, используемые желающими проверить свои знания в игровой форме (так называемые частичные аналоги). 
Выработаны функциональные и нефункциональные требования.

Была разработана архитектура программной системы, для каждой ее составной части было проведено разграничение реализуемых задач проектирование, 
уточнение используемых технологий и непосредственно разработка. Были выбраны наиболее современные средства разработки, широко применяемые в индустрии. 

Полученные в ходе технико-экономического обоснования результаты о прибыли для разработчика, пользователя, уровень рентабельности, а также экономический эффект доказывают целесообразность разработки проекта.